% BACKUP
% FinalReport Priority

\documentclass[conference]{IEEEtran}
\usepackage{graphicx}
\usepackage{cite}
\usepackage{amsmath}
\usepackage{booktabs}
\usepackage{multirow}
\usepackage{array}
\usepackage{float}
\usepackage{algorithmic}
\usepackage{algorithm}
\usepackage[english]{babel}

\title{Automated Tuberculosis Detection from Chest X-Ray Images Using Traditional Digital Image Processing Techniques}

\author{
    \IEEEauthorblockN{Group 5}
    \IEEEauthorblockA{
        Department of Computer Science and Electronics \\
        Universitas Gadjah Mada \\
        Yogyakarta, Indonesia \\
        email@ugm.ac.id
    }
}

\begin{document}

\maketitle

\begin{abstract}
Tuberculosis (TBC)  has become and remains a significant global challenge for thousands of years, causing deaths worldwide, particularly in countries not yet developed. This disease spreads through air, meaning that it is an airborne disease. It spreads through active TBC carrier by coughs or sneeze that releases water droplets containing the bacteria.  It's infectious nature and the fatal potential that it brought if not treated immediately means that fast and accurate early detection becomes the key to decrease mortality rate related to TBC \cite{who2019, sharma2013}. Chest X-ray (CXR) imaging is a widely used method to screen for pulmonary TBC, but it requires manual inference from experts (i.e. doctors) to observe and determine TBC Positivity. And as proven by research, this faces challenges of subjectiveness, variable observer, and time consumption \cite{brady2017, degnan2019}. In addition, TBC radiological patterns often include other lung diseases, and with it potential misclassification \cite{vancleeff2005}. This paper presents an automated TBC detection system utilising traditional digital image processing (DIP) techniques following the AESFERM (Acquisition, Enhancement, Segmentation, Feature Extraction, Representation, Matching) framework. Our approach to this uses Gaussian smoothing, histogram equalisation, and Laplacian sharpening for image enhancement, Otsu thresholding and morphological operations for segmenting the lung, and feature extraction utilising Local Binary Patterns (LBP), Gray-Level-Co-occurence Matrix (GLCM), and HIstogram of Oriented Gradients (HOG). The system returned promising results in distinguishing between normal and chest X-rays affected by TBC, giving an alternative that is less computationally expensive compared to deep learning methods, which gives an advantage for edge devices and resource constraint.
\end{abstract}

\begin{IEEEkeywords}
Tuberculosis Detection, Digital Image Processing, Chest X-Ray Analysis, Computer-Aided Diagnosis, Medical Image Processing
\end{IEEEkeywords}

\section{Introduction}
\label{sec:introduction}

Tuberculosis (TBC) is a serious infectious disease caused by \textit{Mycobacterium tuberculosis}. It has become one of the most significant global health problems for thousands of years \cite{who2019, sharma2013}. TBC remains a leading cause of deaths worldwide, with fast and accurate early detection and treatment crucial for controlling and reduce disease spread and reducing mortality rates \cite{who2019}.

One of the most used method in detection or screening pulmonary TBC is Chest X-ray (CXR)\cite{silverman1949, vanthoog2012}. CXR check allows lung conditions visualisation, helping in identifying earlier signs of TBC infection. In clinical practice, the interpretation of CXR image is usually done by medical personnel or experienced radiologist, which brings its own sets of limitations. These include, but not limited to subjective interpretations, a lot of time, and possibilities for errors or inter-observer opinion variability \cite{brady2017, degnan2019}.

There is more to this challenge. Unfortunately, CXR images of TBC patients often show radiological patterns similar to other lung diseases such as pneumonia or lung cancer. And relying on manual interpretation may frequently lead to misclassification for the reasons earlier. \cite{vancleeff2005}. This situation can have serious consequences as misdiagnosis not only leads to inappropriate therapy but may also worsen health conditions and increase disease transmission risk (do remember that TBC is an airborne disease). The problem is further amplified by the shortage of radiologists, particularly in low-resource developing countries that are most affected by the high prevalence of TBC.

However, with technological advanement, there is hope. In recent years, Computer-Aided Diagnosis (CAD) systems have emerged as promising solutions for assisting medical personnel and radiologist to aid TBC detection based on CXR images. Research by \cite{liu2021} demonstrates that AI-based systems can achieve TBC detection accuracy of 85\%, significantly higher than radiologists without AI assistance (62\%). This brings good news, as the research points out a possibly better ways to detect TBC while simultaneously reducing costs for training new radiologists and subject to less errors. Additionally, several studies have explored various image processing techniques for TBC detection, including hybrid feature approaches \cite{ahmed2023} and chest X-ray enhancement methods \cite{tarambale2021}.

Proving to be substantially better and safer than manual interpretaion, deep learning approaches have shown impressive results. And with it, substantial computational resources and large annotated datasets. And it takes a lot of time, money, and work to improve deep learning models and set everything up for the hospital.

What sets this paper apart is that it explores an alternative approach using traditional digital image processing techniques that offer interpretability, computational efficiency, and effectiveness even with limited data. Our work builds upon existing research in feature representation learning \cite{ghimire2020} and feature detection methods \cite{deepanshu2019}, but focuses specifically on a comprehensive pipeline using well-established image processing algorithms. And of course, the method does not need as much computational power required by deep learning models.

The main contributions of this work include:
\begin{itemize}
\item A complete TBC detection pipeline using traditional digital image processing techniques
\item Implementation of the AESFERM framework specifically tailored for TBC detection in chest X-rays
\item Comprehensive feature extraction combining texture, statistical, and shape information
\item Evaluation of the approach on a publicly available TBC chest X-ray dataset
\end{itemize}

% --------------------------
% BORDER OF EDITING PROGRESS
% --------------------------

\section{Proposed Methodology}
\label{sec:methodology}

Our TB detection system follows the comprehensive AESFERM framework (Acquisition, Enhancement, Segmentation, Feature Extraction, Representation, Matching), as illustrated in Figure \ref{fig:pipeline}. This structured approach ensures systematic processing of chest X-ray images from initial acquisition through final classification.

\subsection{Image Acquisition}
\label{subsec:acquisition}

The initial stage involves acquiring and standardizing chest X-ray images for consistent processing. We utilize the publicly available Tuberculosis Chest X-ray Database from Kaggle \cite{kaggle2018}, which contains 4,200 chest X-ray images (3,500 normal cases, 700 TB cases) in PNG format with 512×512 pixel resolution. 

\textbf{Why this approach:} The dataset provides a substantial number of clinically validated cases with balanced distribution between normal and TB classes. The fixed 512×512 resolution ensures consistency in processing and eliminates variations due to different image dimensions. Grayscale conversion simplifies processing while preserving the essential morphological information needed for TB detection, as color information is redundant in X-ray analysis.

Each image undergoes grayscale conversion to simplify processing while preserving essential morphological information. The standardization process ensures that all input images have consistent dimensions and color space, forming a reliable foundation for subsequent processing stages.

\subsection{Image Enhancement}
\label{subsec:enhancement}

Image enhancement aims to improve visual quality and accentuate features relevant to TB detection. Our enhancement pipeline employs three sequential operations:

\subsubsection{Gaussian Smoothing}
We apply Gaussian smoothing with a 5×5 kernel to reduce random noise inherent in X-ray acquisition:
\begin{equation}
G(x,y) = \frac{1}{2\pi\sigma^2} e^{-\frac{x^2 + y^2}{2\sigma^2}}
\end{equation}
where $\sigma = 0$ for uniform smoothing. This operation preserves important lung structures while minimizing acquisition artifacts.

\textbf{Why this approach:} Gaussian smoothing effectively reduces high-frequency noise without significantly blurring important edges and structures. This is crucial because X-ray images often contain quantum noise from the acquisition process that can interfere with accurate segmentation and feature extraction. The 5×5 kernel size provides optimal balance between noise reduction and detail preservation for 512×512 images.

\subsubsection{Laplacian Sharpening}
Edge enhancement is achieved through Laplacian sharpening:
\begin{equation}
\nabla^2 f = \frac{\partial^2 f}{\partial x^2} + \frac{\partial^2 f}{\partial y^2}
\end{equation}
The sharpening operation emphasizes boundaries and fine details characteristic of TB manifestations, such as cavity walls and lesion contours.

\textbf{Why this approach:} Laplacian sharpening specifically enhances edges and fine details that are critical for identifying TB manifestations like cavitations, infiltrates, and pleural effusions. These features often have subtle intensity transitions that need enhancement for reliable detection. The method works by amplifying high-frequency components where these important diagnostic features reside.

\subsubsection{Histogram Equalization}
Global contrast improvement is performed using histogram equalization, which redistributes intensity values to span the full dynamic range. This enhancement particularly benefits the visualization of subtle TB signs like small infiltrates and early cavitation.

\textbf{Why this approach:} Chest X-rays often suffer from poor contrast due to variations in exposure parameters and patient anatomy. Histogram equalization maximizes the contrast throughout the image, making subtle TB findings more apparent. This is particularly important for early-stage TB where pathological changes may be minimal and poorly contrasted against normal lung tissue.

\subsection{Image Segmentation}
\label{subsec:segmentation}

Accurate lung segmentation is crucial for isolating the region of interest. Our segmentation approach combines multiple techniques:

\subsubsection{Otsu Thresholding}
We employ Otsu's method for automatic optimal threshold calculation, which separates lung tissue from background by minimizing intra-class intensity variance. The resulting binary mask is inverted to position lungs as the foreground.

\textbf{Why this approach:} Otsu's method automatically determines the optimal threshold without manual intervention, making it suitable for automated systems. It works particularly well with chest X-rays because there's typically good separation between the dark lung fields and brighter surrounding tissues. This method handles the bimodal intensity distribution common in chest radiographs effectively.

\subsubsection{Morphological Processing}
Morphological operations refine the initial segmentation:
\begin{itemize}
\item \textbf{Opening} (13×13 elliptical kernel): Removes small noise artifacts and separates connected components
\item \textbf{Closing} (19×19 elliptical kernel): Fills holes in lung parenchyma and smooths boundaries
\end{itemize}

\textbf{Why this approach:} The elliptical kernel shape approximates the natural curvature of lung boundaries better than rectangular kernels. Opening removes small noise particles and separates lungs from other thoracic structures, while closing fills small gaps within the lung fields that may result from blood vessels or pathology. The specific kernel sizes (13×13 and 19×19) were determined empirically to provide optimal performance for 512×512 chest X-rays.

\subsubsection{Connected Component Analysis}
We identify lung regions through connected component analysis with spatial constraints:
\begin{equation}
\text{Lung candidates} = \{C_i | w \times 0.2 < x_{centroid} < w \times 0.8\}
\end{equation}
where only components near the central chest region are considered, and the two largest valid components are selected as left and right lungs.

\textbf{Why this approach:} The spatial constraint (centroid between 20\% and 80\% of image width) ensures that only anatomically plausible lung regions are selected, eliminating artifacts in the periphery. Selecting the two largest components corresponds to the left and right lungs, which are typically the largest dark regions in a chest X-ray. This approach is robust to variations in patient positioning and image quality.

\subsection{Feature Extraction and Representation}
\label{subsec:feature_extraction}

We extract comprehensive features from the segmented lung regions using three complementary approaches inspired by successful implementations in medical image analysis:

\subsubsection{Local Binary Patterns (LBP)}
LBP captures local texture patterns using uniform patterns with $P=8$ and $R=1$:
\begin{equation}
LBP_{P,R} = \sum_{p=0}^{P-1} s(g_p - g_c) \cdot 2^p
\end{equation}
The resulting 59-bin histogram provides rotation-invariant texture descriptors.

\textbf{Why this approach:} LBP effectively captures micro-textural patterns that are characteristic of TB manifestations such as granulomas, cavitations, and fibrotic changes. The uniform pattern variant (P=8, R=1) provides a good compromise between discriminative power and feature dimensionality. LBP's rotation invariance is particularly valuable since TB patterns can appear at various orientations.

\subsubsection{Gray-Level Co-occurrence Matrix (GLCM)}
GLCM extracts statistical texture features across multiple scales and orientations:
\begin{itemize}
\item \textbf{Distances}: [1, 2, 3] pixels for multi-scale analysis
\item \textbf{Angles}: [0, $\pi/4$, $\pi/2$, $3\pi/4$] for orientation invariance
\item \textbf{Features}: Contrast, correlation, energy, homogeneity, ASM, variance, dissimilarity, entropy
\end{itemize}

\textbf{Why this approach:} GLCM provides statistical measures of texture that complement LBP's local patterns. The multiple distances capture texture information at different scales, which is important because TB manifestations vary in size from small nodules to large consolidations. The four angles ensure rotation invariance, while the eight Haralick features provide comprehensive characterization of tissue properties affected by TB, such as heterogeneity (entropy), uniformity (energy), and local variations (contrast).

\subsubsection{Histogram of Oriented Gradients (HOG)}
HOG captures shape and edge information with the following parameters:
\begin{itemize}
\item \textbf{Orientations}: 9 bins
\item \textbf{Pixels per cell}: 8×8
\item \textbf{Cells per block}: 2×2 with L2-Hys normalization
\end{itemize}

\textbf{Why this approach:} HOG effectively captures the structural information and edge patterns that are crucial for identifying TB-related anatomical changes. The 8×8 pixel cells provide sufficient spatial resolution to detect local shape variations, while the 2×2 cell blocks with L2-Hys normalization provide illumination invariance. The 9 orientation bins offer good angular resolution for capturing directional features of TB manifestations such as linear fibrotic streaks or irregular cavity boundaries.

The final feature vector combines all extracted features into a comprehensive 142,990-dimensional representation suitable for machine learning classification. \textbf{Why combined features:} Each feature type captures different aspects of TB pathology—LBP for micro-texture, GLCM for statistical texture, and HOG for shape and edges. Their combination provides a more complete representation of the diverse radiological presentations of tuberculosis.

\section{Experimental Setup and Preliminary Results}
\label{sec:experimental}

\subsection{Dataset and Evaluation Metrics}
\label{subsec:dataset}

% To be filled with experimental details

\subsection{Implementation Details}
\label{subsec:implementation}

% To be filled with implementation details

\subsection{Preliminary Results}
\label{subsec:preliminary}

% To be filled with results

\subsection{Performance Analysis}
\label{subsec:performance_analysis}

% To be filled with analysis

\begin{thebibliography}{00}

\bibitem{who2019}
World Health Organization, ``Global Tuberculosis Report 2019,'' WHO, Geneva, Switzerland, 2019.

\bibitem{sharma2013}
S. K. Sharma and A. Mohan, ``Tuberculosis: From an incurable scourge to a curable disease—journey over a millennium,'' Indian J. Med. Res., vol. 137, no. 3, p. 455, 2013.

\bibitem{silverman1949}
C. Silverman, ``An appraisal of the contribution of mass radiography in the discovery of pulmonary tuberculosis,'' Amer. Rev. Tuberculosis, vol. 60, no. 4, pp. 466–482, 1949.

\bibitem{vanthoog2012}
A. H. van't Hoog, H. K. Meme, K. F. Laserson, J. A. Agaya, B. G. Muchiri, W. A. Githui, L. O. Odeny, B. J. Marston, and M. W. Borgdorff, ``Screening strategies for tuberculosis prevalence surveys: The value of chest radiography and symptoms,'' PLoS ONE, vol. 7, no. 7, Jul. 2012, Art. no. e38691.

\bibitem{brady2017}
A. P. Brady, ``Error and discrepancy in radiology: Inevitable or avoidable?'' Insights Imag., vol. 8, no. 1, pp. 171–182, Feb. 2017.

\bibitem{degnan2019}
A. J. Degnan, E. H. Ghobadi, P. Hardy, E. Krupinski, E. P. Scali, L. Stratchko, A. Ulano, E. Walker, A. P. Wasnik, and W. F. Auffermann, ``Perceptual and interpretive error in diagnostic radiology—Causes and potential solutions,'' Academic Radiol., vol. 26, no. 6, pp. 833–845, Jun. 2019.

\bibitem{vancleeff2005}
M. van Cleeff, L. Kivihya-Ndugga, H. Meme, J. Odhiambo, and P. Klatser, ``The role and performance of chest X-ray for the diagnosis of tuberculosis: A cost-effectiveness analysis in Nairobi, Kenya,'' BMC Infectious Diseases, vol. 5, no. 1, p. 111, Dec. 2005.

\bibitem{liu2021}
Y. Liu, J. Wu, X. Wu, J. Wang, Y. Lu, H. Zhang, and Y. Xu, ``Deep learning assistance for tuberculosis diagnosis with chest radiography in low-resource settings,'' European Respiratory Journal, vol. 58, no. 6, p. 2100633, 2021.

\bibitem{kaggle2018}
Qatar University, University of Dhaka, and University of Malaya, ``Tuberculosis (TBC) Chest X-ray Database,'' Kaggle, 2018. [Online]. Available: https://www.kaggle.com/datasets/tawsifurrahman/tuberculosis-TBC-chest-xray-dataset

\bibitem{ahmed2023}
I. A. Ahmed, E. M. Senan, H. S. A. Shatnawi, Z. M. Alkhraisha, and M. M. A. Al-Azzam, ``Multi-Techniques for Analyzing X-ray Images for Early Detection and Differentiation of Pneumonia and Tuberculosis Based on Hybrid Features,'' Diagnostics, vol. 13, no. 4, art. 814, 2023.

\bibitem{tarambale2021}
M. R. Tarambale and N. S. Lingayat, ``Chest X-ray Enhancement for the Proper Extraction of Suspicious Lung Nodule,'' Int. J. Innov. Eng. Res. Technol., vol. 3, no. 11, pp. 1-9, 2021.

\bibitem{ghimire2020}
S. Ghimire, S. Kashyap, J. T. Wu, A. Karargyris, and M. Moradi, ``Learning Invariant Feature Representation to Improve Generalization across Chest X-ray Datasets,'' arXiv, 2020.

\bibitem{gozes2019}
O. Gozes and H. Greenspan, ``Deep Feature Learning from a Hospital-Scale Chest X-ray Dataset with Application to TBC Detection on a Small-Scale Dataset,'' arXiv, 2019.

\bibitem{deepanshu2019}
Deepanshu T., ``Introduction to Feature Detection and Matching,'' Medium, 2019.

\end{thebibliography}

\end{document}
