% FinalReport Priority

\documentclass[conference]{IEEEtran}
\usepackage{graphicx}
\usepackage{cite}
\usepackage{amsmath}
\usepackage{booktabs}
\usepackage{multirow}
\usepackage{array}
\usepackage{float}
\usepackage{algorithmic}
\usepackage{algorithm}
\usepackage[english]{babel}

\title{Automated Tuberculosis Detection from Chest X-Ray Images Using Traditional Digital Image Processing Techniques}

\author{
    \IEEEauthorblockN{Group 5}
    \IEEEauthorblockA{
        Department of Computer Science and Electronics \\
        Universitas Gadjah Mada \\
        Yogyakarta, Indonesia \\
        email@ugm.ac.id
    }
}

\begin{document}

\maketitle

\begin{abstract}
Tuberculosis (TBC)  has become and remains a significant global challenge for thousands of years, causing deaths worldwide, particularly in countries not yet developed. This disease spreads through air, meaning that it is an airborne disease. It spreads through active TBC carrier by coughs or sneeze that releases water droplets containing the bacteria.  It's infectious nature and the fatal potential that it brought if not treated immediately means that fast and accurate early detection becomes the key to decrease mortality rate related to TBC \cite{who2019, sharma2013}. Chest X-ray (CXR) imaging is a widely used method to screen for pulmonary TBC, but it requires manual inference from experts (i.e. doctors) to observe and determine TBC Positivity. And as proven by research, this faces challenges of subjectiveness, variable observer, and time consumption \cite{brady2017, degnan2019}. In addition, TBC radiological patterns often include other lung diseases, and with it potential misclassification \cite{vancleeff2005}. This paper presents an automated TBC detection system utilising traditional digital image processing (DIP) techniques following the AESFERM (Acquisition, Enhancement, Segmentation, Feature Extraction, Representation, Matching) framework. Our approach to this uses Gaussian smoothing, histogram equalisation, and Laplacian sharpening for image enhancement, Otsu thresholding and morphological operations for segmenting the lung, and feature extraction utilising Local Binary Patterns (LBP), Gray-Level-Co-occurence Matrix (GLCM), and Histogram of Oriented Gradients (HOG). The system returned promising results in distinguishing between normal and chest X-rays affected by TBC, giving an alternative that is less computationally expensive compared to deep learning methods, which gives an advantage for edge devices and resource constraint.
\end{abstract}

\begin{IEEEkeywords}
Tuberculosis Detection, Digital Image Processing, Chest X-Ray Analysis, Computer-Aided Diagnosis, Medical Image Processing
\end{IEEEkeywords}

\section{Introduction}
\label{sec:introduction}

Tuberculosis (TBC) is a serious infectious disease caused by \textit{Mycobacterium tuberculosis}. It has become one of the most significant global health problems for thousands of years \cite{who2019, sharma2013}. TBC remains a leading cause of deaths worldwide, with fast and accurate early detection and treatment crucial for controlling and reduce disease spread and reducing mortality rates \cite{who2019}.

One of the most used method in detection or screening pulmonary TBC is Chest X-ray (CXR)\cite{silverman1949, vanthoog2012}. CXR check allows lung conditions visualisation, helping in identifying earlier signs of TBC infection. In clinical practice, the interpretation of CXR image is usually done by medical personnel or experienced radiologist, which brings its own sets of limitations. These include, but not limited to subjective interpretations, a lot of time, and possibilities for errors or inter-observer opinion variability \cite{brady2017, degnan2019}.

There is more to this challenge. Unfortunately, CXR images of TBC patients often show radiological patterns similar to other lung diseases such as pneumonia or lung cancer. And relying on manual interpretation may frequently lead to misclassification for the reasons earlier. \cite{vancleeff2005}. This situation can have serious consequences as misdiagnosis not only leads to inappropriate therapy but may also worsen health conditions and increase disease transmission risk (do remember that TBC is an airborne disease). The problem is further amplified by the shortage of radiologists, particularly in low-resource developing countries that are most affected by the high prevalence of TBC.

However, with technological advancement, there is hope. In recent years, Computer-Aided Diagnosis (CAD) systems have emerged as promising solutions for assisting medical personnel and radiologist to aid TBC detection based on CXR images. Research by \cite{liu2021} demonstrates that AI-based systems can achieve TBC detection accuracy of 85\%, significantly higher than radiologists without AI assistance (62\%). This brings good news, as the research points out a possibly better ways to detect TBC while simultaneously reducing costs for training new radiologists and subject to less errors. Additionally, several studies have explored various image processing techniques for TBC detection, including hybrid feature approaches \cite{ahmed2023} and chest X-ray enhancement methods \cite{tarambale2021}.

Proving to be substantially better and safer than manual interpretation, deep learning approaches have shown impressive results. And with it, substantial computational resources and large annotated datasets. And it takes a lot of time, money, and work to improve deep learning models and set everything up for the hospital.

What sets this paper apart is that it explores an alternative approach using traditional digital image processing techniques that offer interpretability, computational efficiency, and effectiveness even with limited data. Our work builds upon existing research in feature representation learning \cite{ghimire2020} and feature detection methods \cite{deepanshu2019}, but focuses specifically on a comprehensive pipeline using well-established image processing algorithms. And of course, the method does not need as much computational power required by deep learning models.

The main contributions of this work include:
\begin{itemize}
\item A complete TBC detection pipeline using traditional digital image processing techniques
\item Implementation of the AESFERM framework specifically tailored for TBC detection in chest X-rays
\item Comprehensive feature extraction combining texture, statistical, and shape information
\item Evaluation of the approach on a publicly available TBC chest X-ray dataset
\end{itemize}

\section{Proposed Methodology}
\label{sec:methodology}

The primary constraint that applies in the making of this paper is to stick to traditional digital image processing methods and should not utilise machine learning. The limitation forces the usage of traditional machine learning, and the TBC detection system follows the comprehensive AESFERM framework (Acquisition, Enhancement, Segmentation, Feature Extraction, Representation, Matching). This structured approach ensures systematic processing of chest X-ray images from initial acquisition through final classification. And hopefully, resulting in accurate result that can be used to help medical personnel to better identify early stages of TBC and provide treatment quickly.

\subsection{Image Acquisition}
\label{subsec:acquisition}

Image Acquisition. The stage that starts them all. This stage is the initialisation and involves acquiring and standardising chest X-ray images for consistent processing. For this, we utilise the publicly available Tuberculosis Chest X-ray Database from Kaggle \cite{kaggle2018}. The Kaggle dataset containing 4,200 chest X-ray images (3,500 normal cases, 700 TBC cases) in PNG format with 512×512 pixel resolution which will come in handy, especially consistency.

Why do we use this approach? This Kaggle dataset provides a substantial number of clinically validated cases with balanced distribution between normal and TBC classes. The fixed 512×512 resolution ensures consistency in the processing and eliminates variations that occurs due to different image dimensions. In addition to that, we also use grayscale conversion which simplifies processing while preserving the essential morphological information needed for TBC detection, as color information is redundant in X-ray analysis, due to its grayscale nature.

Before processing the images, each of them undergoes grayscale conversion to simplify processing while preserving essential morphological information. This process of standardisation ensures that all input images have consistent dimensions and color space, thus forming a reliable foundation for the next processing stages.

\subsection{Image Enhancement}
\label{subsec:enhancement}

Having acquired the image, the next step is to perform image enhancement. The Image enhancement stage aims to improve visual quality and highlight features that are relevant to TBC detection. Our enhancement pipeline uses three sequential operations:

\subsubsection{Gaussian Smoothing}
We apply Gaussian smoothing with a 5×5 kernel to reduce random noise inherent in X-ray acquisition:
\begin{equation}
G(x,y) = \frac{1}{2\pi\sigma^2} e^{-\frac{x^2 + y^2}{2\sigma^2}}
\end{equation}
where $\sigma = 0$ for uniform smoothing. This operation makes sure that important lung structures are preserved all while minimizing acquisition artifacts that are present in the picture.

We choose this method, as gaussian smoothing effectively reduces high-frequency noise without significantly blurring important edges and structures. This is really great because X-ray images often contain quantum noise from the acquisition process (the X-ray shot) that can interfere with accurate segmentation and feature extraction. The 5×5 kernel size provides optimal balance between noise reduction and detail preservation for 512×512 images.

\subsubsection{Laplacian Sharpening}
Lung CXR often includes parts that are not used for the machine learning model. We want to focus on the lung part, since it is where the TBC virus lives. Edge detection is used to identify the edges (perimeter) of the lung and feed it for the next stage. Edge enhancement is achieved through Laplacian sharpening:
\begin{equation}
\nabla^2 f = \frac{\partial^2 f}{\partial x^2} + \frac{\partial^2 f}{\partial y^2}
\end{equation}
The sharpening operation emphasizes boundaries and fine details characteristic of TBC manifestations, such as cavity walls and lesion contours. This will aid in the segmentation process later on.

Laplacian sharpening specifically enhances edges and fine details that are critical for identifying TBC-infected lung signs like cavitations, infiltrates, and pleural effusions. These features often have subtle intensity transitions that need enhancement for reliable detection. The method works by amplifying high-frequency components where these important diagnostic features reside.

\subsubsection{Histogram Equalization}
In addition to the two methods above, we also utilise histogram equalisation, which alters contrast. Global contrast improvement is performed by redistributing intensity values to span the full dynamic range (equalisation). This is enhancement particularly beneficial to visualise the subtle TBC signs like small infiltrates and early cavitation.

You see, Chest X-rays often suffer from poor contrast due to variations in exposure parameters and patient anatomy. Histogram equalization maximizes the contrast throughout the image, making subtle TBC findings more apparent. This is particularly important for early-stage TBC where pathological changes may be minimal and poorly contrasted against normal lung tissue.

\subsection{Image Segmentation}
\label{subsec:segmentation}

% --------------------------
% BORDER OF EDITING PROGRESS
% --------------------------

