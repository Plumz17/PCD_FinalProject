% FinalReport v2

\documentclass[conference]{IEEEtran}
\usepackage{graphicx}
\usepackage{cite}
\usepackage{amsmath}
\usepackage{booktabs}
\usepackage{multirow}
\usepackage{array}
\usepackage{float}
\usepackage{algorithmic}
\usepackage{algorithm}
\usepackage[english]{babel}

\title{Automated Tuberculosis Detection from Chest X-Ray Images Using Traditional Digital Image Processing Techniques}

\author{
    \IEEEauthorblockN{Group 5}
    \IEEEauthorblockA{
        Department of Computer Science and Electronics \\
        Universitas Gadjah Mada \\
        Yogyakarta, Indonesia \\
        email@ugm.ac.id
    }
}

\begin{document}

\maketitle

\begin{abstract}
Tuberculosis (TB) remains a significant global health challenge, causing substantial mortality worldwide particularly in developing countries \cite{who2019, sharma2013}. Chest X-ray (CXR) imaging serves as a primary screening tool for pulmonary TB, but manual interpretation faces challenges of subjectivity, inter-observer variability, and time consumption \cite{brady2017, degnan2019}. Furthermore, TB radiological patterns often resemble other lung diseases, leading to potential misclassification \cite{vancleeff2005}. This paper presents an automated TB detection system using traditional digital image processing techniques following the AESFERM framework (Acquisition, Enhancement, Segmentation, Feature Extraction, Representation, Matching). Our approach employs Gaussian smoothing, histogram equalization, and Laplacian sharpening for image enhancement, Otsu thresholding and morphological operations for lung segmentation, and comprehensive feature extraction using Local Binary Patterns (LBP), Gray-Level Co-occurrence Matrix (GLCM), and Histogram of Oriented Gradients (HOG). The system demonstrates promising results in distinguishing between normal and TB-affected chest X-rays, providing a computationally efficient alternative to deep learning methods that is particularly suitable for resource-constrained environments.
\end{abstract}

\begin{IEEEkeywords}
Tuberculosis Detection, Digital Image Processing, Chest X-Ray Analysis, Computer-Aided Diagnosis, Medical Image Processing
\end{IEEEkeywords}

\section{Introduction}
\label{sec:introduction}

Tuberculosis (TB) is a serious infectious disease caused by \textit{Mycobacterium tuberculosis} and has been one of the most significant global health problems for thousands of years \cite{who2019, sharma2013}. According to the World Health Organization, TB remains a leading cause of mortality worldwide, with early detection and treatment being crucial for controlling disease spread and reducing mortality rates \cite{who2019}.

Chest X-ray (CXR) radiography is one of the most frequently used methods for pulmonary TB detection and screening \cite{silverman1949, vanthoog2012}. The method allows visualization of lung conditions, helping identify early signs of TB infection. However, in current clinical practice, CXR image interpretation is typically performed by experienced medical personnel or radiologists, a process that has several significant limitations. These include subjectivity, time consumption, and potential for errors or inter-observer variability \cite{brady2017, degnan2019}.

The challenge is further compounded by the fact that CXR images of TB patients often show radiological patterns similar to other lung diseases such as pneumonia or lung cancer, frequently leading to misclassification \cite{vancleeff2005}. This situation can have serious consequences, as misdiagnosis not only leads to inappropriate therapy but may also worsen health conditions and increase disease transmission risk. The problem is exacerbated by the shortage of radiologists, particularly in low-resource countries that are most affected by high TB prevalence.

In recent years, Computer-Aided Diagnosis (CAD) systems have emerged as promising solutions for assisting TB detection based on CXR images. Research by \cite{liu2021} demonstrates that AI-based systems can achieve TB detection accuracy of 85\%, significantly higher than radiologists without AI assistance (62\%). Additionally, several studies have explored various image processing techniques for TB detection, including hybrid feature approaches \cite{ahmed2023} and chest X-ray enhancement methods \cite{tarambale2021}.

While deep learning approaches have shown impressive results, they often require substantial computational resources and large annotated datasets. This paper explores an alternative approach using traditional digital image processing techniques that offer interpretability, computational efficiency, and effectiveness even with limited data. Our work builds upon existing research in feature representation learning \cite{ghimire2020} and feature detection methods \cite{deepanshu2019}, but focuses specifically on a comprehensive pipeline using well-established image processing algorithms.

The main contributions of this work include:
\begin{itemize}
\item A complete TB detection pipeline using traditional digital image processing techniques
\item Implementation of the AESFERM framework specifically tailored for TB detection in chest X-rays
\item Comprehensive feature extraction combining texture, statistical, and shape information
\item Evaluation of the approach on a publicly available TB chest X-ray dataset
\end{itemize}

\section{Proposed Methodology}
\label{sec:methodology}

Our TB detection system follows the comprehensive AESFERM framework (Acquisition, Enhancement, Segmentation, Feature Extraction, Representation, Matching), as illustrated in Figure \ref{fig:pipeline}. This structured approach ensures systematic processing of chest X-ray images from initial acquisition through final classification.

\subsection{Image Acquisition}
\label{subsec:acquisition}

The initial stage involves acquiring and standardizing chest X-ray images for consistent processing. We utilize the publicly available Tuberculosis Chest X-ray Database from Kaggle \cite{kaggle2018}, which contains 4,200 chest X-ray images (3,500 normal cases, 700 TB cases) in PNG format with 512×512 pixel resolution. 

Each image undergoes grayscale conversion to simplify processing while preserving essential morphological information. The standardization process ensures that all input images have consistent dimensions and color space, forming a reliable foundation for subsequent processing stages.

\subsection{Image Enhancement}
\label{subsec:enhancement}

Image enhancement aims to improve visual quality and accentuate features relevant to TB detection. Our enhancement pipeline employs three sequential operations:

\subsubsection{Gaussian Smoothing}
We apply Gaussian smoothing with a 5×5 kernel to reduce random noise inherent in X-ray acquisition:
\begin{equation}
G(x,y) = \frac{1}{2\pi\sigma^2} e^{-\frac{x^2 + y^2}{2\sigma^2}}
\end{equation}
where $\sigma = 0$ for uniform smoothing. This operation preserves important lung structures while minimizing acquisition artifacts.

\subsubsection{Laplacian Sharpening}
Edge enhancement is achieved through Laplacian sharpening:
\begin{equation}
\nabla^2 f = \frac{\partial^2 f}{\partial x^2} + \frac{\partial^2 f}{\partial y^2}
\end{equation}
The sharpening operation emphasizes boundaries and fine details characteristic of TB manifestations, such as cavity walls and lesion contours.

\subsubsection{Histogram Equalization}
Global contrast improvement is performed using histogram equalization, which redistributes intensity values to span the full dynamic range. This enhancement particularly benefits the visualization of subtle TB signs like small infiltrates and early cavitation.

\subsection{Image Segmentation}
\label{subsec:segmentation}

Accurate lung segmentation is crucial for isolating the region of interest. Our segmentation approach combines multiple techniques:

\subsubsection{Otsu Thresholding}
We employ Otsu's method for automatic optimal threshold calculation, which separates lung tissue from background by minimizing intra-class intensity variance. The resulting binary mask is inverted to position lungs as the foreground.

\subsubsection{Morphological Processing}
Morphological operations refine the initial segmentation:
\begin{itemize}
\item \textbf{Opening} (13×13 elliptical kernel): Removes small noise artifacts and separates connected components
\item \textbf{Closing} (19×19 elliptical kernel): Fills holes in lung parenchyma and smooths boundaries
\end{itemize}

\subsubsection{Connected Component Analysis}
We identify lung regions through connected component analysis with spatial constraints:
\begin{equation}
\text{Lung candidates} = \{C_i | w \times 0.2 < x_{centroid} < w \times 0.8\}
\end{equation}
where only components near the central chest region are considered, and the two largest valid components are selected as left and right lungs.

\subsection{Feature Extraction and Representation}
\label{subsec:feature_extraction}

We extract comprehensive features from the segmented lung regions using three complementary approaches:

\subsubsection{Local Binary Patterns (LBP)}
LBP captures local texture patterns using uniform patterns with $P=8$ and $R=1$:
\begin{equation}
LBP_{P,R} = \sum_{p=0}^{P-1} s(g_p - g_c) \cdot 2^p
\end{equation}
The resulting 59-bin histogram provides rotation-invariant texture descriptors.

\subsubsection{Gray-Level Co-occurrence Matrix (GLCM)}
GLCM extracts statistical texture features across multiple scales and orientations:
\begin{itemize}
\item \textbf{Distances}: [1, 2, 3] pixels for multi-scale analysis
\item \textbf{Angles}: [0, $\pi/4$, $\pi/2$, $3\pi/4$] for orientation invariance
\item \textbf{Features}: Contrast, correlation, energy, homogeneity, ASM, variance, dissimilarity, entropy
\end{itemize}

\subsubsection{Histogram of Oriented Gradients (HOG)}
HOG captures shape and edge information with the following parameters:
\begin{itemize}
\item \textbf{Orientations}: 9 bins
\item \textbf{Pixels per cell}: 8×8
\item \textbf{Cells per block}: 2×2 with L2-Hys normalization
\end{itemize}

The final feature vector combines all extracted features into a comprehensive 142,990-dimensional representation suitable for machine learning classification.

\section{Experimental Setup and Preliminary Results}
\label{sec:experimental}

\subsection{Dataset and Evaluation Metrics}
\label{subsec:dataset}

% To be filled with experimental details

\subsection{Implementation Details}
\label{subsec:implementation}

% To be filled with implementation details

\subsection{Preliminary Results}
\label{subsec:preliminary}

% To be filled with results

\subsection{Performance Analysis}
\label{subsec:performance_analysis}

% To be filled with analysis

\begin{thebibliography}{00}

\bibitem{who2019}
World Health Organization, ``Global Tuberculosis Report 2019,'' WHO, Geneva, Switzerland, 2019.

\bibitem{sharma2013}
S. K. Sharma and A. Mohan, ``Tuberculosis: From an incurable scourge to a curable disease—journey over a millennium,'' Indian J. Med. Res., vol. 137, no. 3, p. 455, 2013.

\bibitem{silverman1949}
C. Silverman, ``An appraisal of the contribution of mass radiography in the discovery of pulmonary tuberculosis,'' Amer. Rev. Tuberculosis, vol. 60, no. 4, pp. 466–482, 1949.

\bibitem{vanthoog2012}
A. H. van't Hoog, H. K. Meme, K. F. Laserson, J. A. Agaya, B. G. Muchiri, W. A. Githui, L. O. Odeny, B. J. Marston, and M. W. Borgdorff, ``Screening strategies for tuberculosis prevalence surveys: The value of chest radiography and symptoms,'' PLoS ONE, vol. 7, no. 7, Jul. 2012, Art. no. e38691.

\bibitem{brady2017}
A. P. Brady, ``Error and discrepancy in radiology: Inevitable or avoidable?'' Insights Imag., vol. 8, no. 1, pp. 171–182, Feb. 2017.

\bibitem{degnan2019}
A. J. Degnan, E. H. Ghobadi, P. Hardy, E. Krupinski, E. P. Scali, L. Stratchko, A. Ulano, E. Walker, A. P. Wasnik, and W. F. Auffermann, ``Perceptual and interpretive error in diagnostic radiology—Causes and potential solutions,'' Academic Radiol., vol. 26, no. 6, pp. 833–845, Jun. 2019.

\bibitem{vancleeff2005}
M. van Cleeff, L. Kivihya-Ndugga, H. Meme, J. Odhiambo, and P. Klatser, ``The role and performance of chest X-ray for the diagnosis of tuberculosis: A cost-effectiveness analysis in Nairobi, Kenya,'' BMC Infectious Diseases, vol. 5, no. 1, p. 111, Dec. 2005.

\bibitem{liu2021}
Y. Liu, J. Wu, X. Wu, J. Wang, Y. Lu, H. Zhang, and Y. Xu, ``Deep learning assistance for tuberculosis diagnosis with chest radiography in low-resource settings,'' European Respiratory Journal, vol. 58, no. 6, p. 2100633, 2021.

\bibitem{kaggle2018}
Qatar University, University of Dhaka, and University of Malaya, ``Tuberculosis (TB) Chest X-ray Database,'' Kaggle, 2018. [Online]. Available: https://www.kaggle.com/datasets/tawsifurrahman/tuberculosis-tb-chest-xray-dataset

\bibitem{ahmed2023}
I. A. Ahmed, E. M. Senan, H. S. A. Shatnawi, Z. M. Alkhraisha, and M. M. A. Al-Azzam, ``Multi-Techniques for Analyzing X-ray Images for Early Detection and Differentiation of Pneumonia and Tuberculosis Based on Hybrid Features,'' Diagnostics, vol. 13, no. 4, art. 814, 2023.

\bibitem{tarambale2021}
M. R. Tarambale and N. S. Lingayat, ``Chest X-ray Enhancement for the Proper Extraction of Suspicious Lung Nodule,'' Int. J. Innov. Eng. Res. Technol., vol. 3, no. 11, pp. 1-9, 2021.

\bibitem{ghimire2020}
S. Ghimire, S. Kashyap, J. T. Wu, A. Karargyris, and M. Moradi, ``Learning Invariant Feature Representation to Improve Generalization across Chest X-ray Datasets,'' arXiv, 2020.

\bibitem{gozes2019}
O. Gozes and H. Greenspan, ``Deep Feature Learning from a Hospital-Scale Chest X-ray Dataset with Application to TB Detection on a Small-Scale Dataset,'' arXiv, 2019.

\bibitem{deepanshu2019}
Deepanshu T., ``Introduction to Feature Detection and Matching,'' Medium, 2019.

\end{thebibliography}

\end{document}